\documentclass[10pt,a4paper,oneside]{report}
\usepackage[utf8]{inputenc}
\usepackage[german]{babel}
\usepackage[T1]{fontenc}
\usepackage{amsmath}
\usepackage{amsfonts}
\usepackage{amssymb}
\usepackage[left=2cm,right=2cm,top=2cm,bottom=2cm]{geometry}

\usepackage{amsthm}
\usepackage{graphicx}

\newtheorem{theorem}{Satz}[chapter]
\newtheorem{corollary}{Korollar}[chapter]
\newtheorem{lemma}{Lemma}[chapter]
\newtheorem{hlemma}{Hilfslemma}[chapter]
%\theoremstyle{definition}
\newtheorem{definition}{Definition}[chapter]
\newtheorem{beispiel}{Beispiel}[chapter]

\author{Ludwig Paul Lind}
\title{Ecken ABC-Flächen}
\begin{document}
\maketitle
\tableofcontents

\chapter{Einleitung}
kurze Vorstellung ABC-Flächen. CAD-Systeme usw. 
Meine Untersuchung Ecken blablabla
Einteilung der Kapitel

\chapter{Mathematische Grundlagen}
Grundlagen finite Elemente.
Problem der DGL

\chapter{ABC-Flächen}
Arbeit von Uli im wesentlichen zusammentragen.
G0 und G1 Bedingung herausstellen. 
Vllt mit mehr Bildern als im Paper.
Anwendung nicht so wichtig.

\chapter{Fragestellung}
Betrachte Ecken. Genügen weniger scharfe Bedingungen. Inspiration woher, für diese Annahme?

\chapter{Minimalbeispiel}
Abbildung graphisch darstellen. 
Am Ende Resultate zusammentragen und Frage aufwerfen, ob dies verallgemeinerbar ist

\chapter{Verallgemeinerung}
Existenz des Diffeomorphismuses
Erhalt der Eigenschaften als ABC-Fläche bei Verkettung mit Diffeomorphismus
Folgerung der Verallgemeinerung

\chapter{Bedeutung der Resultate}
WIP

\chapter{Fazit}

\chapter{Anhang}
Rechnungen bezüglich Minimalbeispiel und Integrabilität bis $L^{2-\epsilon}$

\end{document}
