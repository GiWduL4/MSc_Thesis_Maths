\documentclass[10pt,a4paper,oneside]{report}
\usepackage[utf8]{inputenc}
\usepackage[german]{babel}
\usepackage[T1]{fontenc}
\usepackage{amsmath}
\usepackage{amsfonts}
\usepackage{amssymb}
\usepackage[left=2cm,right=2cm,top=2cm,bottom=2cm]{geometry}

\usepackage{amsthm}
\usepackage{graphicx}
\usepackage{mathtools}

\newtheorem{theorem}{Satz}[chapter]
\newtheorem{corollary}{Korollar}[chapter]
\newtheorem{lemma}{Lemma}[chapter]
\newtheorem{hlemma}{Hilfslemma}[chapter]
%\theoremstyle{definition}
\newtheorem{definition}{Definition}[chapter]
\newtheorem{beispiel}{Beispiel}[chapter]

\newcommand{\R}{\mathbb{R}}
\newcommand{\OL}{\mathcal{O}}

\author{Ludwig Paul Lind}
\title{Ecken ABC-Flächen}
\begin{document}
\maketitle
\tableofcontents

\chapter{Einleitung}
kurze Vorstellung ABC-Flächen. CAD-Systeme usw. 
Verweis Uli
Meine Untersuchung Ecken blablabla
Einteilung der Kapitel

\chapter{Mathematische Grundlagen}
Grundsätzlich Thema DGL
Beispiel für DGL -u'' = f
Grundlagen finite Elemente.
Voraussetzungen
Sobolevspaces/schwache Ableitungen usw.

\chapter{ABC-Flächen}
Motivation
Arbeit von Uli im wesentlichen zusammentragen.
G0 Bedingung herausstellen. 
Mit mehr Bildern als im Paper.
Anwendung nicht so wichtig.

\chapter{Fragestellung}
In ihrer Arbeit beschreiben Florian Martin und Ulrich Reif Bedingungen, um Stetigkeit, Normalenstetigkeit und Krümmungsstetigkeit in den Ecken garantieren zu können.
Außerhalb der Ecken ist die Glattheit trivial.

In dieser Arbeit soll im wesentlichen darauf eingegangen werden unter welchen Bedingungen auf den ABC-Flächen welcher Differentialgleichungstyp lösbar ist.
Für Differentialgleichungen erster Ordnung ist die G1-Bedingung genügend, weil die partiellen Ableitungen in diesem Fall beschränkt sind.
Es stellt sich nun die Frage, ob bereits weniger scharfe Bedingungen Lösbarkeit auf der ABC-Fläche ermöglichen oder gar implizieren.

Inspiration woher, für diese Annahme?

Betrachte zunächst als Motivation ein relativ simples Beispiel, das die G0-Bedingung erfüllt.
\begin{beispiel} \label{bsp:G0pos}
Es wird eine Ecke einer ABC-Fläche betrachtet, die über einem rechtwinkligen Definitionsgebiet definiert ist.
Es sei $\Gamma = \left[0,\alpha\right]^2$ das Definitionsgebiet, in dem die Umgebung der im Ursprung liegenden Ecke untersucht werden soll.
Die beiden Kanten des Definitionsgebiets $\Gamma$ werden entsprechend Abbildung \ref{fig:bsp:defgbt} mit $\Gamma_0$ und $\Gamma_1$ bezeichnet.
Beschreibung hier gut und ausführlich. 
Als eigenes Besipiel herausstellen

Bild \label{fig:bsp:defgbt}

Die Gewichte seien von der Gestalt
\begin{align*}
w_0\left(x,y\right) &= y \cdot p\left(x,y\right) \\
w_1\left(x,y\right) &= x \cdot q\left(x,y\right) \\
w_j\left(x,y\right) &= xy \cdot s_j\left(x,y\right) \quad \forall j \neq 0,1 \\
w\left(x,y\right) &= xy \cdot t\left(x,y\right).
\end{align*}
Dabei seien $p, q, s_j, t$ Funktionen, die hinreichend glatt sowie beschränkt sind und im Ursprung positiven Wert haben.
Hier genauer werden, mit Landausymbolen und so
Außerdem darauf eingehen, dass damit auch sinx = x * sinx/x gemeint sein kann
Weiterhin muss die Bedingung der Kontaktordnung überprüft werden

Zur Vereinfachung der Notation wird der Ausdruck
\begin{align*}
\bar{w}\left(x,y\right) = w\left(x,y\right) + \sum_{j \neq 0,1} w_j\left(x,y\right) = xy \cdot r\left(x,y\right)
\end{align*}
verwendet, wobei die Funktion $r = \sum_{j \neq 0,1} s_j + t$ die Eigenschaften von $s_j$ und $t$ erbt.

Damit ist die ABC-Fläche
\begin{align*}
\bold{a} = \frac{\bold{b} w + \sum_{j \neq 0,1} \bold{\bar{r}}_j w_j + \bold{\bar{r}}_0 w_0 + \bold{\bar{r}}_1 w_1}{\bar{w} + w_0 +w_1}
\end{align*}
für noch festzulegende Bases und Ribbons definiert.
Es lässt sich nun zeigen, dass, sofern diese die G0 Bedingung erfüllen, das heißt, dass die Ribbons in der Ecke konsistent im Sinne $\bar{r}_0\left(0,0\right) = \bar{r}_1\left(0,0\right)$ sind, die partiellen Ableitungen von $\bold{a}$ beschränkt sind.

Betrachte die offensichtlich beschränkte Funktion 
\begin{align*}
\varphi\left(x,y\right) = \frac{w\left(x,y\right)}{\bar{w}\left(x,y\right) + w_0\left(x,y\right)+ w_1 \left(x,y\right)} = \frac{xy\cdot t\left(x,y\right)}{xy\cdot r\left(x,y\right) + x\cdot q\left(x,y\right)+ y \cdot p \left(x,y\right)}
\end{align*}
Als partielle Ableitung ergibt sich
\begin{align*}
\varphi_{x} = \frac{x^2y^2 \left(t_x r - t r_x\right)+ x^2 y \left(t_x q -t q_x\right) + x y^2 \left( t_x p - t p_x\right)+ y^2 t p}{\left(xy r + x q+ y p \right)^2}.
\end{align*}
Zur besseren Lesbarkeit sind die Variablen nicht mit notiert und die partielle Ableitung $\partial_x$ wird durch den Index kenntlich gemacht.
Die Ableitung $\varphi_{x}$ ist im Ursprung nicht stetig fortsetzbar.
So ist 
\begin{align*}
\lim_{x \to 0} \varphi \left(x, 0\right) = \lim_{x \to 0} 0 = 0, 
\end{align*}
während
\begin{align*}
\lim_{y \to 0} \varphi \left(0, y\right) = \lim_{y \to 0} \frac{t}{p}= \frac{t}{p} \neq 0,
\end{align*}
nicht verschwindet.
Die Funktion $\varphi_{x}$ ist mit beliebiger Fortsetzung im Ursprung aber fast überall beschränkt und es gilt somit
\begin{align*}
\varphi_{x} \in L^{\infty}.
\end{align*}
Aus der Symmetrie folgt 
\begin{align*}
\varphi_{y} \in L^{\infty}.
\end{align*}

Für die Funktionen
\begin{align*}
\varphi_j\left(x,y\right) = \frac{w_j\left(x,y\right)}{\bar{w}\left(x,y\right) + w_0\left(x,y\right)+ w_1 \left(x,y\right)} 
\end{align*}
folgt analog in den Fällen $j \neq 0,1$, dass
\begin{align*}
\varphi_{j x}, \varphi_{j y} \in L^{\infty}
\end{align*}
gilt.
Hieraus folgt wegen
\begin{align*}
\varphi_0 + \varphi_1 = \frac{w_0+w_1}{\bar{w}+ w_0 + w_1} = 1 - \frac{w + \sum_{j\neq 0,1} w_j}{\bar{w}+ w_0 + w_1}
\end{align*}
sofort, dass
\begin{align*}
\varphi_{0 x} + \varphi_{1 x}, \varphi_{0 y} + \varphi_{1 y} \in L^{\infty}
\end{align*}
gilt.

Nun wird die gesamten ABC-Fläche $\bold{b} \varphi + \sum_{j\neq 0,1} \bold{\bar{r}}_j \varphi_j + \bold{\bar{r}}_0 \varphi_0 + \bold{\bar{r}}_1 \varphi_1$ untersucht und ihre partiellen Ableitungen betrachtet.
\begin{align*}
\bold{a}_x = \bold{b}_x \varphi + \bold{b} \varphi_x + \sum_{j\neq 0,1} \bold{\bar{r}}_{jx} \varphi_j + \sum_{j\neq 0,1} \bold{\bar{r}}_j \varphi_{jx} + \bold{\bar{r}}_{0x} \varphi_0 \bold{\bar{r}}_{0} \varphi_{0x}+ \bold{\bar{r}}_{1x} \varphi_1 + \bold{\bar{r}}_{1} \varphi_{1x} 
\end{align*}
Mit der Glattheit der Base und der Ribbons und den bisher gezeigten Eigenschaften liegt dieser bis auf zwei Summanden in $L^{\infty}$.
Der zu untersuchende Rest lautet
\begin{align*}
g \coloneqq \bold{\bar{r}}_{0} \varphi_{0x} + \bold{\bar{r}}_{1} \varphi_{1x}
\end{align*}
Aufgrund der Konsistenz der Ribbons gilt $\bold{r}_0 \left(\tau\right) = \bold{r}_1 \left(0,0\right) + \mathcal{O} \left( \left| \tau \right| \right)$.
Damit ist
\begin{align*}
g = \bold{\bar{r}}_{1} \left( \varphi_{0x} + \varphi_{1x} \right) + \varphi_{0x} \cdot \mathcal{O} \left( \left| \tau \right| \right).
\end{align*}
Mit  
\begin{align*}
\varphi_{0x} = \frac{x y^2 \left( p_xr- p r_x\right) + xy \left(p_x q -p q_yx\right) - y^2 pr - ypq}{\left(xy r + x q+ y p \right)^2}
\end{align*}
ist direkt ersichtlich, dass jeder Summand der Ausdrücke $x \cdot \varphi_{0x}$ und $y \cdot \varphi_{0x}$ in der Umgebung des Ursprungs mit beliebiger Fortsetzung im Ursprung fast überall beschränkt ist. 
Somit ist gezeigt, dass $g \in L^{\infty}$ gilt, woraus direkt
\begin{align*}
\bold{a} \in W^{1,\infty}
\end{align*}
folgt.
\end{beispiel}

Beispiel \ref{bsp:G0pos} zeigt, dass bei geschickter Wahl der Gewichte die Lösbarkeit von Differentialgleichungen zweiter Ordnung bereits unter den G0-Bedingungen erreicht werden kann.
Es handelt sich bei dem gezeigten Fall bereits um eine allgemeine Formulierung, dies so in vielen Anwendungen Gebrauch finden kann.

Dass diese Bedingungen alleine aber nicht genügen zeigt folgendes Beispiel.
(mehr Text)

\begin{beispiel}
In analoger Weise zu obigem Beispiel wird wieder eine im Ursprung befindliche rechtwinklige Ecke einer ABC-Fläche betrachtet.
Diesmal seien die Gewichte zu
\begin{align*}
w_0 &= y^3 \\
w_1 &= x \\
w &= xy^3
\end{align*}
gewählt.
Die Wahl der Base und der Ribbons fällt auf 
\begin{align*}
\bold{b} &= 0 \\
\bold{r}_0 &= y \\
\bold{r}_1 &= 0 
\end{align*}
blablabla
\end{beispiel}

Die Integrierbarkeit ist also nicht trivialerweise durch die Glattheitsbedingung G0 gegeben.
Von Interesse für Anwender der ABC-Flächen ist jetzt welche Anforderungen insbesondere bei der Wahl der Gewichte erfüllt werden müssen, um Integrierbarkeit zu erhalten.

\chapter{Allgemeine Betrachtung}

Zunächst soll abstrahiert von den gezeigten Beispielen, auf die in den Kapiteln ??? erneut eingegangen werden wird, untersucht werden, aus welchen Gründen bei ABC-Flächen in der Ecke überhaupt Integrierbarkeit nicht allgemein vorliegt.

Ableitungen von a sind beschränkt außer der Dh*(r1-r0) Teil
\begin{align*}
\bold{a} &= \frac{w\bold{b} + w_0 \bold{r}_0+ w_1 \bold{r}_1}{w + w_0 + w_1} \\
&= \frac{w}{w+w_0+w_1} \bold{b} + \frac{w_0 + w_1}{w+w_0+w_1} \bold{r}_0 + \frac{w_1}{w+w_0+w_1} \left(\bold{r}_1-\bold{r}_0\right)
\end{align*}


Der Term kann aber auch explodieren

Auch Normalenstetigkeit aus Paper geht kaputt

Konkretes Beispiel mit den qs aus dem Paper

Für praktische Anwendungen sind am ehesten die Gewichte zu beeinflussen.
Deshalb sollte es das Ziel sein, für diese eine Vorschrift zu finden, die $L^p$-Integrierbarkeit garantiert.

w = w0 w1 q

wj entspricht w für j nicht 0, 1

\chapter{Höhenfunktionen in rechtwinkliger Geometrie}

In diesem Kapitel soll die Problemstellung, die in den beiden Beispielen des Kapitels \ref{chap:fragestellung} vorgestellt wurde, allgemeiner gefasst werden.

Es wird eine ABC-Fläche 
\begin{align*}
\bold{a}: \Gamma \mapsto \R^3
\end{align*}
betrachtet, wobei wieder blablabla
(jeweils nur z-Komponente (Höhenfunktion))

Bild

Es wird zunächst der polynomielle Fall betrachtet

Gewichte
\begin{align*}
w_0 &= y^m \\
w_1 &= x^n \\
w &= x^n y^m
\end{align*}
Betrachte den Fall
\begin{align*}
r_1 - r_0 = x^k
\end{align*}
Damit folgt für den kritischen Term der ABC-Fläche
\begin{align*}
g = \frac{x^k x^n}{x^n y^m + x^n + y^m}
\end{align*}
Betrachte partielle Ableitungen
\begin{align*}
g_x = \frac{\left(k+n\right) x^{n+k-1} \left( x^ny^m+x^n+y^m\right) - x^{n+k} n x^{n-1} \left(y^m + 1\right)}{\left(x^ny^m+x^n+y^m\right)^2}
\end{align*}
Für $k \geq 1$ ist $a_x \in L^{\infty}$ klar.
Dagegen liegt 
\begin{align*}
a_y &= \frac{-x^{n+k} m y^{m-1} \left(x^n + 1\right)}{\left(x^ny^m+x^n+y^m\right)^2} \\
&=  \frac{-my^{m-1}x^{2n+k}}{\left(x^ny^m+x^n+y^m\right)^2} - \frac{my^{m-1}x^{n+k}}{\left(x^ny^m+x^n+y^m\right)^2}
\end{align*}
für beliebiges $p$ nicht in $L^p$, wenn $n$ groß genug gewählt wird.
Dies lässt sich wie folgt zeigen: 
\begin{align*}
&\int_{x=0}^{1} \int_{y=0}^{1} \left( \frac{y^{m-1}x^{n+k}}{\left(x^ny^m+x^n+y^m\right)^2} \right)^p \, \mathrm{d}x \, \mathrm{d}y \\
\geq \, &C \int_{x=0}^{1} \int_{y=0}^{1}  \frac{y^{p\left(m-1\right)}x^{p\left(n+k\right)}}{\left(x^{2n}+y^{2m}\right)^p}  \, \mathrm{d}x \, \mathrm{d}y \\
\geq \, &C \int_{x=0}^{1} \int_{y=0}^{x^{\frac{n}{m}}}  \frac{y^{p\left(m-1\right)}x^{p\left(n+k\right)}}{\left(x^{2n}+y^{2m}\right)^p}  \, \mathrm{d}x \, \mathrm{d}y \\
\geq \, &C' \int_{x=0}^{1} \int_{y=0}^{x^{\frac{n}{m}}}  \frac{y^{\left(m-1\right)p}x^{p\left(n+k\right)}}{x^{2np}}  \, \mathrm{d}x \, \mathrm{d}y \\
= \, &C' \int_{x=0}^{1} \frac{x^{kp}}{x^{np}} \int_{y=0}^{x^{\frac{n}{m}}} y^{mp-p}  \, \mathrm{d}x \, \mathrm{d}y \\
= \, &C'' \int_{x=0}^{1} \frac{x^{kp}}{x^{np}} \left[y^{mp-p+1}\right]_{0}^{x^{\frac{n}{m}}} \, \mathrm{d}x\\
= \, &C'' \int_{x=0}^{1} \frac{1}{x^{\left(n-k\right)p}} x^{\frac{n}{m}\left(mp-p+1\right)} \, \mathrm{d}x\\
\end{align*}
Das Integral divergiert für den Fall
\begin{align*}
p \left(n - k\right) &\geq \frac{n}{m}\left(mp-p+1\right) + 1 \\
\frac{n}{m} &\geq \frac{\left(1+pk\right)}{p-1}
\end{align*}
Bei der Untersuchung auf Quadratintegrabilität ($p=2$) können die Gibbons bis zu beliebiger Ordnung $k$ übereinstimmen und dennoch liegt $a_y$ für hinreichend großes $n$ nicht in $L^2$.

Fall $p = \infty$ gesondert betrachten. Da wird das größer benötigt.
Tippen morgen früh

Die oben gefundene Schranke ist scharf, wie folgende Rechnung zeigt
\begin{align*}
&\int_{x=0}^{1} \int_{y=0}^{1} \left( \frac{y^{m-1}x^{n+k}}{\left(x^n+y^m\right)^2} \right)^p \, \mathrm{d}x \, \mathrm{d}y \\
\leq &\int_{x=0}^{1} \int_{y=0}^{1} \frac{y^{\left(m-1\right)p}x^{\left(n+k\right)p}}{\left(x^{2n}+y^{2m}\right)^p}  \, \mathrm{d}x \, \mathrm{d}y 
\end{align*}
Substitution $z^n = y^m$ mit $\frac{\mathrm{d}z}{\mathrm{d}y} = \frac{m}{n} y^{\frac{m}{n} - 1}$.
\begin{align*}
&\int_{x=0}^{1} \int_{z=0}^{1} \frac{z^{\frac{n}{m}\left[\left(m-1\right)p - \left(\frac{m}{n} - 1\right)\right]}x^{\left(n+k\right)p}}{\left(x^{2n}+z^{2n}\right)^p}  \, \mathrm{d}x \, \mathrm{d}z \\
= &\int_{x=0}^{1} \int_{z=0}^{1} \frac{z^{np-\frac{n}{m}p-1+\frac{n}{m}}x^{\left(n+k\right)p}}{\left(x^{2n}+z^{2n}\right)^p}  \, \mathrm{d}x \, \mathrm{d}z \\
\end{align*}
Wechsel in Polarkoordinaten
\begin{align*}
x &= r \cos \left( \varphi \right) \\
z &= r \sin \left( \varphi \right)
\end{align*}
Damit ergibt sich 
\begin{align*}
&\leq \int_{\varphi=0}^{\frac{\pi}{2}} \int_{r=0}^{\sqrt{2}} \frac{\left(r \cos\left(\varphi\right)\right)^{np-\frac{n}{m}p-1+\frac{n}{m}}\left(r \sin\left(\varphi\right)\right)^{\left(n+k\right)p}}{\left(\left(r \sin\left(\varphi\right)\right)^{2n}+\left(r \cos\left(\varphi\right)\right)^{2n}\right)^p}  r \, \mathrm{d}r \, \mathrm{d}\varphi \\
&= \int_{\varphi=0}^{\frac{\pi}{2}} 
\frac{\left( \cos \left(\varphi\right)\right)^{np-\frac{n}{m}p-1+\frac{n}{m}} 
\left( \sin \left(\varphi\right)\right)^{\left(n+k\right)p}}
{\left(\left(\sin\left(\varphi\right)\right)^{2n}+\left(\cos\left(\varphi\right)\right)^{2n}\right)^p} \, \mathrm{d}\varphi 
\int_{r=0}^{\sqrt{2}} \frac{r^{np-\frac{n}{m}p-1+\frac{n}{m}+np+kp}}{r^{2np}}  r \, \mathrm{d}r \\
\end{align*}
Das erste Integral existiert für $p\geq 1$, weil der Nenner positiv beschränkt ist.
Das zweite Integral konvergiert unter der Bedingung
\begin{align*}
np-\frac{n}{m}p-1+\frac{n}{m}+np+kp + 1 &> 2np-1 \\
\frac{n}{m} &< \frac{1+kp}{p-1}
\end{align*}

Analoges gilt für r1-r0 = $y^k$

Allgemeinere Fassung mit weiteren Fkt q und p usw

Resultat herausstellen

An anderer Stelle nochmal auf die Darstellung x * q(x,y) eingehen und Beispiele zeigen, dass so beliebige Fkt mit O(tau) und glatt so darstellbar sind.

\chapter{Verallgemeinerung auf beliebige Definitionsgebiete}
Höhenfunktionsergebnisse verallgemeinern auf 3D-Funktionen

Existenz des Diffeomorphismuses

Erhalt der Eigenschaften als ABC-Fläche bei Verkettung mit Diffeomorphismus
Folgerung der Verallgemeinerung

wieder in D
laut Uli gut darstellbar

\chapter{Bedeutung der Resultate}
WIP
Bilder
Und auch Bedeutung für Krümmung

Zweite Ableitungen hier verpacken

Insbesondere auf physikalische Bedeutung der lösbaren DGLs eingehen. Welche DGL Typen sind dann lösbar

\chapter{Fazit}
Ausblick Krümmung


\chapter{Anhang}
Rechnungen bezüglich Minimalbeispiel und Integrabilität bis $L^{2-\epsilon}$ zB

\chapter{Betrachtung einer rechtwinkligen Ecke}
Kann weg


Abbildung graphisch darstellen. 

Es wird die Abbildung $\bold{a}$: $\left[0,1\right]^2 \to \R^3$ betrachtet, deren Funktionsvorschrift 
\begin{align*}
a\left(u,v\right) = \frac{\bold{b} w + \bold{r}_1 w_1 + \bold{r}_2 w_2}{w+w_1+w_2}
\end{align*}
lautet.

Betrachte Ecke einer ABC-Fläche mit Definitionsgebiet $\left[0,1\right]^2$.
G0 Bedingung erfüllt

Betrachte w1, w2, w.
wj erfüllt Voraussetzungen von w außerhalb der Ecke. Deshalb nicht zu berücksichtigen.

Wir wissen, dass für $\sigma = \gamma_1\left(u\right) = \left(u,0\right)$ der Zusammenhang
\begin{align*}
\frac{w\left(\sigma + \tau\right)}{w_1\left(\sigma + \tau\right)} \in \OL\left(|\tau|\right)
\end{align*}
gilt.
Wegen des Spezialfalls $\tau = \left( 0, v\right)^{\top}$ muss somit
\begin{align*}
\frac{w\left(u,v\right)}{w_1\left(u,v\right)} \in \OL\left(v\right)
\end{align*}
gelten.
Es folgt
\begin{align*}
q\left(u,v\right) \coloneqq \frac{w\left(u,v\right)}{v \cdot w_1\left(u,v\right)} \in \OL\left(1\right).
\end{align*}
Die Funktion ist somit beschränkt in einer Umgebung $U$ um den Ursprung. 
Aus der geforderten Glattheit der Gewichte, folgt somit, dass die Funktion $q$ außer in der Ecke glatt ist auf $U$.
Es lässt sich also schreiben:
\begin{align*}
w\left(u,v\right) = v w_1\left(u,v\right) q\left(u,v\right)
\end{align*}
mit $q \in C^\infty\left(U\right)$.

Analog lässt sich der Zusammenhang
\begin{align*}
w\left(u,v\right) = u w_0\left(u,v\right) p\left(u,v\right)
\end{align*}
mit $p \in C^\infty\left(U\right)$ finden.

Im folgenden soll nun die Funktion
\begin{align*}
h\left(u,v\right) = \frac{w}{w + w_0 + w_1}
\end{align*}
untersucht werden.
\begin{align*}
\partial_u h &= \frac{\partial_u w \left(w+w_0+w_1\right) - w \left( \partial_u w + \partial_u w_0 +\partial_u w_1\right)}{\left(w+w_0+w_1\right)^2} \\
&= \frac{\partial_u w w_0+ \partial_u w w_1 - w \partial_u w_0 - w \partial_u w_1}{\left(w+w_0+w_1\right)^2}
\end{align*}
Mit den obigen Umschreibungen der Funktion $w$ und deren Ableitung ergibt sich
\begin{align*}
\partial_u w w_0 - w \partial_u w_0 &= \left(w_0 q + u \partial_u w_0 q +u w_0 \partial_u q \right) w_0 - u w_0 q \partial_u w_0 \\
&= w_0^2 \left(q+u\partial_u q\right)
\end{align*}
sowie 
\begin{align*}
\partial_u w w_1 - w \partial_u w_1 &= \left(v \partial_u w_1 p + v w_1 \partial_u p\right) w_1 - v w_1 p \partial_u w_1 \\
&= w_1^2 v \partial_u p
\end{align*}

Die Beschränkheit von $\partial_u h$ ist somit gezeigt, während die Beschränkheit von $\partial_v h$ analog folgt. 
Damit ist klar, dass
\begin{align*}
h \in W^{1, \infty} \supset H^1
\end{align*}
gilt.

Für die Funktion 
\begin{align*}
h' = 1 - h = \frac{w_0 + w_1}{w + w_0 +w_1}
\end{align*}
folgt auch direkt die Zugehörigkeit zum Sobolevraum $W^{1, \infty}$

Die Funktion 
\begin{align*}
\varphi_0 = \frac{w_0}{w + w_0 + w_1}
\end{align*}
ist beschränkt und bis auf ihm Ursprung stetig.
Mit Satz ??? folgt somit, dass die Funktionen $u \partial_u \varphi_0\left(u,v\right)$ und $v \partial_u \varphi_0\left(u,v\right)$ beschränkt in dem Sinne sind, dass sie in $L^{\infty}$ liegen.

Zusammentragen, dass
a in W1, inf liegt

Am Ende Resultate zusammentragen und Frage aufwerfen, ob dies verallgemeinerbar ist



\end{document}
