\documentclass[10pt,a4paper]{report}
\usepackage[utf8]{inputenc}
\usepackage[german]{babel}
\usepackage[T1]{fontenc}
\usepackage{amsmath}
\usepackage{amsfonts}
\usepackage{amssymb}
\usepackage[left=2cm,right=2cm,top=2cm,bottom=2cm]{geometry}

\usepackage{float}
\usepackage{subfig}

\author{Ludwig Paul Lind}
\begin{document}
\chapter{Hilfsrechnungen}
Betrachte
\begin{align*}
f&: \left[0,1\right]^2 \rightarrow \mathbb{R} \\
f& \left(x, y\right) = \frac{x^a y^b}{\left(xy+x+y\right)^2}.
\end{align*}
Es gilt:
\begin{table}[H]
\centering
\begin{tabular}{|c||c|c|c|}
\hline 
$a+b$ & $\geq 2$ & $1$ & $0$ \\ 
\hline 
$f \in$ & $L^{\infty}$ & $L^{2-\epsilon}$ & $L^{1-\epsilon}$ \\ 
\hline 
\end{tabular} 
\end{table}

Entsprechend ist für die Funktion
\begin{align*}
f&: \left[0,1\right]^2 \rightarrow \mathbb{R} \\
f& \left(x, y\right) = \frac{x^a y^b}{\left(xy+x+y\right)^3}
\end{align*}
die Zuordnung
\begin{table}[H]
\centering
\begin{tabular}{|c||c|c|c|}
\hline 
$a+b$ & $\geq 3$ & $2$ & $1$ \\ 
\hline 
$f \in$ & $L^{\infty}$ & $L^{2-\epsilon}$ & $L^{1-\epsilon}$ \\ 
\hline 
\end{tabular}
\end{table} 
zu finden.

Analog gilt für
\begin{align*}
f&: \left[0,1\right]^2 \rightarrow \mathbb{R} \\
f& \left(x, y\right) = \frac{x^a y^b}{\left(x^2y^2+x^2+y^2\right)^2}
\end{align*}
die Einteilung:
\begin{table}[H]
\centering
\begin{tabular}{|c||c|c|c|}
\hline 
$a+b$ & $\geq 4$ & $3$ & $2$ \\ 
\hline 
$f \in$ & $L^{\infty}$ & $L^{2-\epsilon}$ & $L^{1-\epsilon}$\\ 
\hline 
\end{tabular}
\end{table} 

Analog gilt für
\begin{align*}
f&: \left[0,1\right]^2 \rightarrow \mathbb{R} \\
f& \left(x, y\right) = \frac{x^a y^b}{\left(x^2y^2+x^2+y^2\right)^3}
\end{align*}
die Einteilung:
\begin{table}[H]
\centering
\begin{tabular}{|c||c|c|c|}
\hline 
$a+b$ & $\geq 6$ & $5$ & $4$ \\ 
\hline 
$f \in$ & $L^{\infty}$ & $L^{2-\epsilon}$ & $L^{1-\epsilon}$\\ 
\hline 
\end{tabular}
\end{table} 

\chapter{Minimalbeispiele}
Hier Minimalbeispiel beschreiben mit $\Gamma_1$ auf x-Achse und $\Gamma_2$ auf y-Achse
\section{Kontaktordnung 0}
Wir wählen
\begin{align*}
w &= xy \\
w_1 &= x \\
w_2 &= y.
\end{align*}
\subsection{Ribbons und Normale konsistent}
Einfachste Annahme ist, dass $b$ und $r_1 = r_2 = r$ konstant sind.
Damit folgt für die ABC-Fläche
\begin{align*}
f \left(x, y\right) = \frac{bxy + rx + ry}{xy+x+y}.
\end{align*}
Diese Funktion ist offensichtlich beschränkt.

Betrachte nun die partiellen Ableitungen.
\begin{align*}
\partial_x f \left(x, y\right) = \frac{\left(b-r\right) y^2}{\left( xy+x+y\right)^2} \in L^{\infty}
\end{align*}
Zweite Ableitungen
\begin{align*}
\partial_x \partial_x f \left(x, y\right) = \frac{-2 \left(b-r\right) y^2 (y+1)}{\left( xy+x+y\right)^3} \in L^{2-\epsilon} \\
\partial_y \partial_x f \left(x, y\right) = \frac{2 \left(b-r\right) xy}{\left( xy+x+y\right)^3} \in L^{2-\epsilon}
\end{align*}

\subsection{Ribbons konsistent aber Normale nicht}
Nehme an, $b$ ist konstant und $r_1 \circ \kappa_1 = y+c$ und $r_2 \circ \kappa_2 = x+c$.
Für die ABC-Fläche gilt dann
\begin{align*}
f \left(x, y\right) &= \frac{bxy + \left(y+c\right)x + \left(x+c\right)y}{xy+x+y} \\
&= \frac{\left(b+2\right)xy + cx + cy}{xy+x+y}.
\end{align*}
Die Betrachtung ist also analog zu vorigem Fall.

\section{Kontaktordnung 1}
Wir wählen
\begin{align*}
w &= x^2y^2 \\
w_1 &= x^2 \\
w_2 &= y^2.
\end{align*}
\subsection{Ribbons und Normale konsistent}
Einfachste Annahme ist, dass $b$ und $r_1 = r_2 = r$ konstant sind.
Damit folgt für die ABC-Fläche
\begin{align*}
f \left(x, y\right) = \frac{bx^2y^2 + rx^2 + ry^2}{x^2y^2+x^2+y^2}.
\end{align*}
Diese Funktion ist offensichtlich beschränkt.

Betrachte nun die partiellen Ableitungen.
\begin{align*}
\partial_x f \left(x, y\right) = \frac{\left(2b-2r\right) xy^4}{\left(x^2y^2+x^2+y^2\right)^2} \in L^{\infty}
\end{align*}

Zweite Ableitungen
\begin{align*}
\partial_x \partial_x f \left(x, y\right) = \frac{\left(2b-2r\right) y^4 \left(y^2-3x^2y^2-3x^2\right)}{\left(x^2y^2+x^2+y^2\right)^3} \in L^{\infty} \\
\partial_y \partial_x f \left(x, y\right) = \frac{\left(8b-8r\right) x^3y^3}{\left(x^2y^2+x^2+y^2\right)^3} \in L^{\infty}
\end{align*}

\subsection{Ribbons konsistent aber Normale nicht}
Nehme an, $b$ ist konstant und $r_1 \circ \kappa_1 = y+c$ und $r_2 \circ \kappa_2 = x+c$.
Für die ABC-Fläche gilt dann
\begin{align*}
f \left(x, y\right) &= \frac{bx^2y^2 + \left(y+c\right)x^2 + \left(x+c\right)y^2}{x^2y^2+x^2+y^2} \\
&= \frac{bx^2y^2 + cx^2 + cy^2}{x^2y^2+x^2+y^2} + \frac{yx^2+xy^2}{x^2y^2+x^2+y^2}.
\end{align*}
Nur der rechte Summand ist interessant, da er die Abweichung vom vorigen Fall beschreibt.
\begin{align*}
g \left(x, y\right) = \frac{yx^2+xy^2}{x^2y^2+x^2+y^2}
\end{align*}
Betrachte Ableitungen von $g$.
\begin{align*}
\partial_x g \left(x, y\right) = \frac{2xy^3 + x^2y^4 + x^2y^2+y^4}{\left(x^2y^2+x^2+y^2\right)^2} \in L^{\infty}
\end{align*}
Zweite Ableitungen
\begin{align*}
\partial_x \partial_x g \left(x, y\right) = \frac{2y^5-6x \left(y^6+y^4\right)-6x^2\left(y^5+y^3\right)+2x^3\left(y^4+2y^4+y^2\right)}{\left(x^2y^2+x^2+y^2\right)^3} \in L^{2-\epsilon} \\
\partial_y \partial_x g \left(x, y\right) = \frac{6\left(x^2y^3+x^3y^2\right)-2\left(x^4y^3+x^4y+x^3y^4+xy^4\right)}{\left(x^2y^2+x^2+y^2\right)^3} \in L^{2-\epsilon}
\end{align*}
\end{document}