\documentclass[10pt,a4paper]{report}
\usepackage[utf8]{inputenc}
\usepackage[german]{babel}
\usepackage[T1]{fontenc}
\usepackage{amsmath}
\usepackage{amsfonts}
\usepackage{amssymb}
\usepackage[left=2cm,right=2cm,top=2cm,bottom=2cm]{geometry}

\usepackage{float}
\usepackage{subfig}

\author{Ludwig Paul Lind}
\begin{document}
\chapter{Beispiel}
Gewichte
\begin{align*}
w_0 &= y^2 \\
w_1 &= x^n \\
w &= x^n y^2
\end{align*}
Bases und Ribbons (jeweils nur z-Komponente (Höhenfunktion))
\begin{align*}
b &= 0 \\
r_0 &= 0 \\
r_1 &= x^k
\end{align*}
Damit folgt für die ABC-Fläche
\begin{align*}
a = \frac{x^k x^n}{x^n y^2 + x^n + y^2}
\end{align*}
Betrachte partielle Ableitungen
\begin{align*}
a_x = \frac{\left(k+n\right) x^{n+k-1} \left( x^ny^2+x^n+y^2\right) - x^{n+k} n x^{n-1} \left(y^2 + 1\right)}{\left(x^ny^2+x^n+y^2\right)^2}
\end{align*}
Für $k \geq 1$ ist $a_x \in L^{\infty}$ klar.
Dagegen liegt 
\begin{align*}
a_y &= \frac{-x^{n+k} 2 y \left(x^n + 1\right)}{\left(x^ny^2+x^n+y^2\right)^2} \\
&=  \frac{-2yx^{2n+k}}{\left(x^ny^2+x^n+y^2\right)^2} - \frac{2yx^{n+k}}{\left(x^ny^2+x^n+y^2\right)^2}
\end{align*}
für beliebiges $p$ nicht in $L^p$, wenn $n$ groß genug gewählt wird.
Dies lässt sich wie folgt zeigen:
\begin{align*}
&\int_{x=0}^{1} \int_{y=0}^{1} \left( \frac{yx^{n+k}}{\left(x^ny^2+x^n+y^2\right)^2} \right)^p \, \mathrm{d}x \, \mathrm{d}y \\
\geq \, &C \int_{x=0}^{1} \int_{y=0}^{1}  \frac{y^px^{p\left(n+k\right)}}{\left(x^{2n}+y^4\right)^p}  \, \mathrm{d}x \, \mathrm{d}y \\
\geq \, &C \int_{x=0}^{1} \int_{y=0}^{x^{\frac{n}{2}}}  \frac{y^px^{p\left(n+k\right)}}{\left(x^{2n}+y^4\right)^p}  \, \mathrm{d}x \, \mathrm{d}y \\
\geq \, &C' \int_{x=0}^{1} \int_{y=0}^{x^{\frac{n}{2}}}  \frac{y^px^{p\left(n+k\right)}}{x^{2np}}  \, \mathrm{d}x \, \mathrm{d}y \\
= \, &C' \int_{x=0}^{1} \frac{x^{kp}}{x^{np}} \int_{y=0}^{x^{\frac{n}{2}}} y^p  \, \mathrm{d}x \, \mathrm{d}y \\
= \, &C'' \int_{x=0}^{1} \frac{x^{kp}}{x^{np}} \left[y^{p+1}\right]_{0}^{x^{\frac{n}{2}}} \, \mathrm{d}x\\
= \, &C'' \int_{x=0}^{1} \frac{1}{x^{\left(n-k\right)p}} x^{\frac{n}{2}\left(p+1\right)} \, \mathrm{d}x\\
\end{align*}
Das Integral divergiert für den Fall
\begin{align*}
p \left(n - k\right) &\geq \frac{n}{2}\left(p+1\right) + 1 \\
n &\geq \frac{2 \left(1+pk\right)}{p-1}
\end{align*}
Bei der Untersuchung auf Quadratintegrabilität ($p=2$) können die Ribbons bis zu beliebiger Ordnung $k$ übereinstimmen und dennoch liegt $a_y$ für hinreichend großes $n \geq 4k+2$ nicht in $L^2$.
\end{document}